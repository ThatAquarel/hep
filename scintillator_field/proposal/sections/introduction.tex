At any given moment, there are muons raining down on us at a rate of approximately 1 muon per square centimeter per minute [1]. Traditionally, they are observed with the use of a spark chamber. However, building a spark chamber is a technical challenge, especially for high school students. We wanted to create a simple and easily reproducible way of observing muons, which is what led us to develop our three-dimensional scintillator-based detector, a.k.a. The Scintillating Chamber. Although we have developed the detector’s software and hardware components, precise calibration would require exposure to a controlled muon beam. Access to CERN’s world-class facilities would allow us to validate our detector’s performance and refine its accuracy. It would also provide invaluable insights into innovative techniques that would contribute to the broader field of particle detection technology. We hope to collaborate with peers and experts at CERN to deepen our comprehension of physics concepts through real-world applications. We are also committed to inspiring other young Canadian scientists by sharing our experiences and findings through outreach activities. An opportunity to conduct our experiment at CERN would be a pivotal step in this journey. We hope that our design can be used in physics classrooms around the world to assist science education and communication. Our team hopes to continue development to enhance our detector for these purposes, adding a timing device to measure the energy of the detected particles, and creating a better user interface for easier use in classrooms.
